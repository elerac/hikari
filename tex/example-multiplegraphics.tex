\documentclass{article}
\usepackage[dvipdfmx]{graphicx}
\usepackage[subrefformat=parens]{subcaption}
\usepackage{multiplegraphics}

\title{Example of multiplegraphics.sty}
\author{elerac}
\date{\today}

\usepackage{natbib}
\usepackage{graphicx}

\begin{document}

\maketitle

\section{Usage}

\subsection{Simple}

Fig.\ref{fig:myfigures} shows multiple images arranged horizontally. 
In particular, Fig.\ref{fig:myfigures}\subref{fig:name1} shows "figures/image1.png".

\begin{figure}[htbp]
    \newcommand{\figuresinfo}{
    {figures/image1.png}/{name1}/{fig:name1},
    {figures/image2.png}/{name2}/{fig:name2},
    {figures/image3.png}/{name3}/{fig:name3}} % {file name}/{subcaption name}/{label name}
    
    \includemultiplegraphics{\figuresinfo}
    \caption{Figures}
    \label{fig:myfigures}
\end{figure}

\subsection{Scaling}

Scale the figure by half.

\begin{figure}[htbp]
    \newcommand{\figuresinfo}{
    {figures/image1.png}/{name1}/{fig:name1-half},
    {figures/image2.png}/{name2}/{fig:name2-half},
    {figures/image3.png}/{name3}/{fig:name3-half}}
    
    \includemultiplegraphics[0.5]{\figuresinfo}
    \caption{Figures}
\end{figure}

\newpage

\subsection{Matrix}
Arrange the figures in a matrix.

\begin{figure}[htbp]
    \includemultiplegraphics[0.9]{{
    {figures/image1.png}/{name1}/{fig:name1-mat},
    {figures/image2.png}/{name2}/{fig:name2-mat},
    {figures/image3.png}/{name3}/{fig:name3-mat}}}\vspace{3pt}
    
    \includemultiplegraphics[0.9]{{
    {figures/image1.png}/{name4}/{fig:name4-mat},
    {figures/image2.png}/{name5}/{fig:name5-mat},
    {figures/image3.png}/{name6}/{fig:name6-mat}}}\vspace{3pt}
    
    \includemultiplegraphics[0.9]{{
    {figures/image1.png}/{name7}/{fig:name7-mat},
    {figures/image2.png}/{name8}/{fig:name8-mat},
    {figures/image3.png}/{name9}/{fig:name9-mat}}}
    
    \caption{Figures}
    \label{fig:myfigures-mat}
\end{figure}

\end{document}